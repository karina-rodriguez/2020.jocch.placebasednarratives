%%%%%%%%%%%%%%%%%%%%%%%%%%%%%%%%%%%%%%%%%%%%%%%%%%%%%%%%%%%%%%
%
% Tim's Mark-up Macros
%
% To remove all subsequent inline comments, place at the beginning of your document:
%
%      HERE, OSKAR: =============.
%                                 |
%                                 |
%\def\hidecomments{}  %  <========'
%
% By undefining hidecomments again, comments can be temporarily re-enabled.
%
% require:
%
%  \usepackage[usenames,dvipsnames]{xcolor}
%  \usepackage[normalem]{ulem}  % for strike-through (\sout)
%
% usage:
%
%  \TWedit{text insertiong}
%  \TWedit[removed this text]{replaced by this one}
%  \TW{comment}
%
% Alternative definitions of the underlying ``EDIT'' can be used for
% collaborative writing, display of diffs, final version for approval,
% etc.  Note that for the final version, author identifiers disappear
% and color coding changes to displaying of removals and insertions.
%
%
\def\clap#1{\hbox to 0pt{\hss #1\hss}}%
\def\initials#1{\protect\clap{\smash{\raisebox{1.4ex}{\tiny{\textsf{\textit{#1}}}}}}}%
%
\makeatletter
\newcommand{\NOTE}[3]{\protect\@ifundefined{hidecomments}{%
  \strut{\color{#2}{\hspace{0pt}\initials{#1}\protect{[#3]}}}%
  }{}}
\newcommand{\EDITbyauthor}[4][]{\protect\@ifundefined{hidecomments}{%
  \strut{\color{#3}{\hspace{0pt}\initials{#2}\protect\sout{#1}{#4}}}%
  }{}}
\newcommand{\EDITredandgreen}[4][]{\protect\@ifundefined{hidecomments}{%
  \strut{\color{Red}{\hspace{0pt}\protect\sout{#1}{\color{ForestGreen}{#4}}}}%
  }{}}
\newcommand{\EDITgreenonly}[4][]{\protect\@ifundefined{hidecomments}{%
  \strut{\color{ForestGreen}{#4}}%
  }{}}
\newcommand{\EDITfinal}[4][]{\protect%
  \strut{#4}%
  {}}
\newcommand{\NOTEboxed}[3]{\protect\@ifundefined{hidecomments}{%
  {\centering\fbox{\parbox{0.97\linewidth}{\protect\EDIT{#1}{#2}{#3}}}}%
  }{}}

%\newcommand{\EDIT}[4][]{\EDITredandgreen[#1]{#2}{#3}{#4}}
\newcommand{\EDIT}[4][]{\EDITgreenonly[#1]{#2}{#3}{#4}}
%\newcommand{\EDIT}[4][]{\EDITbyauthor[#1]{#2}{#3}{#4}}
%\newcommand{\EDIT}[4][]{\EDITfinal[#1]{#2}{#3}{#4}}

\newcommand{\greenBegin}{\color{ForestGreen}}
\newcommand{\greenEnd}{\color{black}}
\newcommand{\redStart}{\color{Red}}
\newcommand{\redEnd}{\color{black}}

\makeatother
%
\newcommand{\TWedit}[2][]{\protect\EDIT[#1]{TW}{Orange}{#2}}
\newcommand{\TW}[1]{\protect\NOTE{TW}{Orange}{#1}}
\newcommand{\TWboxed}[1]{\NOTEboxed{TW}{Orange}{#1}}
%
\newcommand{\KRedit}[2][]{\protect\EDIT[#1]{KR}{blue}{#2}}
\newcommand{\KR}[1]{\protect\NOTE{KR}{blue}{#1}}
\newcommand{\KRboxed}[1]{\NOTEboxed{KR}{blue}{#1}}
%
\newcommand{\MSedit}[2][]{\protect\EDIT[#1]{MS}{green}{#2}}
\newcommand{\MS}[1]{\protect\NOTE{MS}{green}{#1}}
\newcommand{\MSboxed}[1]{\NOTEboxed{MS}{green}{#1}}
%
%
%%%%%%%%%%%%%%%%%%%%%%%%%%%%%%%%%%%%%%%%%%%%%%%%%%%%%%%%%%%%%%

\def\ignore#1{}

\usepackage{bm}
% see also: http://tex.stackexchange.com/questions/3238/bm-package-versus-boldsymbol

%\newcommand{\vect}[1]{\textbf{#1}}      % use US standard more common in graphics
\newcommand{\vect}[1]{{\ensuremath{\boldsymbol{#1}}}}    % use ISO standard (doesn't seem to work with default font, but does with \usepackage{newtxmath})
%\newcommand{\vect}[1]{\textbf{\textit{#1}}}    % use ISO standard (doesn't work for Greek symbols)
%\newcommand{\vect}[1]{\textbf{\textit{#1}}}    % use ISO standard (doesn't work for Greek symbols)
%\newcommand{\vect}[1]{{\ensuremath{\bm{#1}}}} % use ISO standard (occasionally falls back to Knuth's font)

%\newcommand{\mtrx}[1]{#1}
\newcommand{\mtrx}[1]{\vect{#1}}

%\newcommand{\trans}{^{\!\top}\!}
\newcommand{\trans}{^{\top}\!}  % new math font does better kerning

% already defined in our-commands.tex
%\DeclareMathOperator*{\argmin}{arg\,min}

\newcommand{\styleMap}[1]{\mathcal{#1}}

%\newcommand{\styleModelPointVect}[1]{\texttt{p.#1}}
%\newcommand{\styleModelPointScal}[1]{\texttt{p.#1}}
%\newcommand{\styleModelPointVect}[1]{\vect{#1}^\text{m}}
%\newcommand{\styleModelPointScal}[1]{     {#1}^\text{m}}
\newcommand{\styleModelPointVect}[1]{\bar{\vect{#1}}}
\newcommand{\styleModelPointScal}[1]{\bar{     {#1}}}


%\newcommand{\myparagraph}[2]{\paragraph{#1}}
\newcommand{\myparagraph}[2][\quad]{\smallskip\noindent\mbox{\textbf{#2}}#1}
%\newcommand{\myparagraph}[2][\quad]{\smallskip\noindent\mbox{\sffamily\textsc{\textbf{#2}}}#1}

\newcommand{\RR}{{\mathbb R}}

%\newcommand{\weight}{\omega}   % reads like ``oeight''
\newcommand{\weight}{w}         % reads like ``weight''

%\def\twsubsubsectionstar#1{\vspace{1.5ex}\noindent\textsf{\MakeUppercase{#1}}\par\nopagebreak\vspace{0.5ex}\nopagebreak\noindent}
\def\twsubsubsectionstar#1{\vspace{2ex}\noindent\textsf{#1}\par\nopagebreak\vspace{1ex}\nopagebreak\noindent}

\DeclareMathOperator*{\argmax}{argmax}
\DeclareMathOperator*{\argmin}{argmin}

% Lots of figures is OK
\renewcommand{\textfraction}{0}%{0.05}
\renewcommand{\topfraction}{1}%{0.95}
\renewcommand{\bottomfraction}{1}%{0.95}
\renewcommand{\dbltopfraction}{1}%{0.9}
\renewcommand{\floatpagefraction}{1}%{0.95}
